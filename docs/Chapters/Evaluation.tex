\section{Evaluation}
\label{sec:evaluation}

We test our implementation by playing against an agent who uses the naive heuristic and a search depth of seven.
We evaluate all three heuristics with time limits 2, 4, 6, 8 and 10 seconds.
The given time limit is strictly adhered to, the average time per move over a game is about 0.2 seconds lower than the limit (e.g., 5.8 s for a limit of 6 s).

\subsection{Naive Heuristic}

As can be seen in~\cref{tab:naive}, the naive heuristic is sufficient to win against the other agent, even though they use the same heuristic.
This is mainly due to the fact that we use iterative deepening and thus reach higher search depths.
Our implementation can reach depths ten (2 s) or eleven (10 s).

\begin{table}[!ht]
\centering
\begin{tabular}{ |c||c|c|c||c|c|c| } 
 \hline
 Time & Player & Won & Score & Player & Won & Score \\ \hline \hline
  2 & White & Yes & 16 & Black & Yes & -16 \\ 
  4 & White & Yes & 36 & Black & Yes & -35 \\  
  6 & White & Yes & 36 & Black & Yes & -48 \\  
  8 & White & Yes & 38 & Black & Yes & -56 \\   
 10 & White & Yes & 64 & Black & Yes & -56 \\ 
 \hline
\end{tabular}
\caption{\label{tab:naive} End scores for different time limits when playing as \textit{white} or \textit{black} using the naive heuristic.}
\end{table}

This heuristic is good when a high search depth can be reached.
In this case, the heuristic is evaluated on positions which are terminal positions or close to one.
However, if the search can not go deep and is evaluated on earlier positions of the board, the heuristic value is not reliable, since the board changes a lot in the course of the game.
If a player has a lot of stones in the beginning this does not necessarily mean that they will have a lot of stones in the end.


\subsection{Mobility Heuristic}
\cref{tab:mobil} shows the results when using the mobility heuristic.
This heuristic is more computationally complex than the naive heuristic, therefore only depths eight and nine could be reached.
This heuristic is a lot better for \textit{black} than for \textit{white}, which shows that this heuristic is not reliable on its own.

\begin{table}[!ht]
\centering
\begin{tabular}{ |c||c|c|c||c|c|c| }
 \hline
 Limit & Player & Won & Score & Player & Won & Score \\ \hline \hline
  2 & White & Yes &  30 & Black & No  & 2 \\ 
  4 & White & No  & -41 & Black & Yes & -12 \\  
  6 & White & No  & -12 & Black & Yes & -50 \\  
  8 & White & No  & -14 & Black & Yes & -50 \\   
 10 & White & No  & -13 & Black & Yes & -50 \\ 
 \hline
\end{tabular}
\caption{\label{tab:mobil} End scores for different time limits when playing as \textit{white} or \textit{black} using the mobility heuristic.}
\end{table}

Opposed to the naive heuristic, the mobility heuristic is assumed to be better for earlier positions.
In the beginning, it may be better to try and restrict the opponents possibilities.
However, the quality of the moves is of more importance than their quantity which is why this heuristic does not yield good results.


\subsection{Compound Heuristic}
The best results were achieved with the compound heuristic, even though only depths eight end nine could be reached.
As can be seen in~\cref{tab:comp}, when playing as \textit{white} we win with more than sixty points for every time limit and with more than forty points when playing as \textit{black}.
Despite the fact that the mobility heuristic on its own is not satisfactory, it improves the naive heuristic when they are used together, as the reached scores are better.

\begin{table}[!ht]
\centering
\begin{tabular}{ |c||c|c|c||c|c|c| } 
 \hline
 Limit & Player & Won & Score & Player & Won & Score \\ \hline \hline
  2 & White & Yes & 62 & Black & Yes & -42 \\ 
  4 & White & Yes & 64 & Black & Yes & -45 \\  
  6 & White & Yes & 62 & Black & Yes & -46 \\  
  8 & White & Yes & 62 & Black & Yes & -62 \\   
 10 & White & Yes & 64 & Black & Yes & -61 \\ 
 \hline
\end{tabular}
\caption{\label{tab:comp} End scores for different time limits when playing as \textit{white} or \textit{black} using the compound heuristic.}
\end{table}

An agent using this heuristic uses the best of $h_{n}$ and $h_{m}$.
In the beginning, when the board is almost empty, the agent tries to restrict the other player's possibilities.
With time, while the board is being filled, the agent shifts it's focus to yielding a high score.
Hence, the mobility becomes negligible, while the importance of the amount of stones on the board increases.
